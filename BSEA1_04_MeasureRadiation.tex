\documentclass{jarticle}

\usepackage[dvipdfmx]{graphicx}
\usepackage{float}
\usepackage{url}
\usepackage{caption}
% \captionsetup[table]{justification=centering}

\title{電気回路}
\author{2511198 肥田幸久}
\date{2025年11月11日作成}

\begin{document}
\maketitle



\section{実験の目的}

本実験では, 電気回路の中で起こる力学的な現象を観測し, 減衰振動の周波数やピーク電圧と時間の関係といった量を求めることを目的とする.



\section{実験の原理}

抵抗値が$R$の抵抗器の両端にかかる電圧を$V_R(t)$, 流れる電流を$I(t)$とすると, オームの法則により次式が成り立つ.
\begin{equation}
V_R(t) = R I(t)
\end{equation}
コンデンサは充電時間を除けば絶縁体である.
コンデンサの静電容量を$C$とすると, ある時間にコンデンサに蓄えられている電荷$q(t)$とコンデンサの両端にかかる電圧$V_C(t)$の関係は次式で表される.
\begin{equation}
  q(t) = C V_C(t)
\end{equation}
電流とは単位時間あたりに流れる電荷の量であるため, 次式が成り立つ.
\begin{equation}
  I(t) = \frac{d q(t)}{d t}
\end{equation}
コイルは時間変化する電流に対して誘導起電力を発生させる.
コイルの自己インダクタンスを$L$, ある時間にコイルを流れる電流を$I(t)$とすると, 誘導起電力$V_L(t)$は次式で表される.
\begin{equation}
  V_L(t) = -L \frac{d I(t)}{d t}
\end{equation}

\begin{figure}[H]
  \centering
  \includegraphics[width=0.6\linewidth]{picture_principle.png}
  \caption{RLC直列回路の模式図}
  \label{fig:RLC_series_circuit}
\end{figure}


\section{実験方法}





\section{実験結果}





\section{考察}





\section{まとめ}



\begin{thebibliography}{99}


\end{thebibliography}

\end{document}