\documentclass{jarticle}

\usepackage[dvipdfmx]{graphicx}
\usepackage{float}
\usepackage{url}
\usepackage{caption}
\usepackage{amsmath}
% \captionsetup[table]{justification=centering}

\title{電気回路}
\author{2511198 肥田幸久}
\date{2025年11月11日作成}

\begin{document}
\maketitle



\section{実験の目的}

本実験では, 電気回路の中で起こる力学的な現象を観測し, 減衰振動の周波数やピーク電圧と時間の関係といった量を求めることを目的とする.



\section{実験の原理}

抵抗値が$R$の抵抗器の両端にかかる電圧を$V_R(t)$, 流れる電流を$I(t)$とすると, オームの法則により次式が成り立つ.
\begin{equation}
V_R(t) = RI(t)
\end{equation}
コンデンサは充電時間を除けば絶縁体である.
コンデンサの静電容量を$C$とすると, ある時間にコンデンサに蓄えられている電荷$q(t)$とコンデンサの両端にかかる電圧$V_C(t)$の関係は次式で表される.
\begin{equation}
  q(t) = CV_C(t)
\end{equation}
電流とは単位時間あたりに流れる電荷の量であるため, 次式が成り立つ.
\begin{equation}
  I(t) = \frac{dq(t)}{dt}
  \label{eq:current_charge}
\end{equation}
コイルは時間変化する電流に対して誘導起電力を発生させる.
コイルの自己インダクタンスを$L$, ある時間にコイルを流れる電流を$I(t)$とすると, 誘導起電力$V_L(t)$は次式で表される.
\begin{equation}
  V_L(t) = -L \frac{dI(t)}{dt}
\end{equation}

\begin{figure}[H]
  \centering
  \includegraphics[width=0.6\linewidth]{picture_principle.png}
  \caption{RLC直列回路の模式図}
  \label{fig:RLC_series_circuit}
\end{figure}

図\ref{fig:RLC_series_circuit}の回路に電池をつないだ瞬間に回路を流れる電流を考える.

電池の起電力を$V$とし, ある時刻にコンデンサに蓄えられている電荷を$q(t)$, 抵抗値, インでクタンス, コンデンサの静電容量をそれぞれ$R$, $L$, $C$とすると, 回路に流れる電流$I(t)$は次式で表される.
\begin{equation}
  V - L \frac{dI(t)}{dt} = RI(t) + \frac{q(t)}{C}
\end{equation}
両辺を時間で微分し, 式\ref{eq:current_charge}を用いて, $2\gamma=R/L$, $\omega_0^2 = 1/(LC)$とおくと, 次式が得られる.
\begin{equation}
  \frac{d^2I(t)}{dt^2} + 2 \gamma\frac{dI(t)}{dt} + \omega_0^2 I(t) = 0
\end{equation}

この式は$I(t)$を$x(t)$と置き換えると, 減衰振動を表す微分方程式と同じ形になり, 解は$\gamma$と$\omega_0$の大小関係によって次の3通りに分類される.

\vskip\baselineskip
(i) $\gamma < \omega_0$(減衰振動)
\begin{equation}
  I(t) = e^{-\gamma t} (a \sin{\omega_1 t} + b \cos{\omega_1 t})
\end{equation}

(ii) $\gamma > \omega_0$(過減衰)
\begin{equation}
  I(t) = e^{-\gamma t} (ae^{\omega_1 t} + be^{-\omega_1 t})
\end{equation}

(iii) $\gamma = \omega_0$(臨界減衰)
\begin{equation}
  I(t) = e^{-\gamma t} (a t + b)
\end{equation}

ここで, $a$, $b$はいずれの場合も初期条件によって決まる定数, 角周波数$\omega_1$は過減衰の場合$\omega_1 = \sqrt{\gamma^2 - \omega_0^2}$, 減衰振動の場合$\omega_1 = \sqrt{\omega_0^2 - \gamma^2}$である.
実際に図\ref{fig:RLC_series_circuit}の回路を組み立て, 時間と電流の関係を観測すると, 減衰振動, 過減衰, 臨界減衰のいずれの現象も確認でき, 角周波数$\omega_1$や抵抗係数$\gamma$を求めることができる.



\section{実験方法}

直流回路において臨界減衰を観測することは困難であるため, 本実験では減衰振動とか減衰について実験を行った.
回路は$R_0$の抵抗を用いた際に臨界減衰が起こるようなインダクタンス$L$と静電容量$C$を選び, 抵抗値$R$を変えることで減衰振動や過減衰が起こるようにした.
回路の電源として発振器を接続し, 電流の時間変化を直接観測するのではなく, 抵抗感電圧$V_R(t)$をオシロスコープで観測した.

はじめに, コイルのインでクタンス$L$をLCRメータで, コンデンサの静電容量$C$と抵抗値$R$をマルチメータで測定した.
次に, 次式を用いて, 抵抗値の平均$R_0$と$L$から, 臨界減衰($\gamma = \omega_0$)が起こるコンデンサの静電容量$C_0$を計算した.
$2\gamma = R/L$, $\omega_0^2 = 1/(LC)$, $\gamma = \omega_0$より,
\begin{eqnarray*}
  \frac{R}{2L} &=& \sqrt{\frac{1}{LC_0}} \\
  \frac{R^2}{4L^2} &=& \frac{1}{LC_0} \\
  C_0 &=& \frac{4L}{R^2}
\end{eqnarray*}
測定した宣伝容量$C$が$C_0$に最も近い値になるようにコンデンサを選んだ.

これらのコイルとコンデンサ, 抵抗器を用い, 与えられた4つの抵抗を順番に繋いで回路を組み立て測定を行った.



\section{実験結果}





\section{考察}





\section{まとめ}



\begin{thebibliography}{99}


\end{thebibliography}

\end{document}