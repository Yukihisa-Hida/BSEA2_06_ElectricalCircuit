\documentclass{jarticle}

\usepackage[dvipdfmx]{graphicx}
\usepackage{float}
\usepackage{url}
\usepackage{caption}
\usepackage{amsmath}
\captionsetup[table]{justification=centering}

\title{電気回路}
\author{2511198 肥田幸久}
\date{2025年11月11日作成}

\begin{document}
\maketitle



\section{実験の目的}

本実験では, 電気回路の中で起こる力学的な現象を観測し, 減衰振動の周波数やピーク電圧と時間の関係といった量を求めることを目的とする.



\section{実験の原理}

抵抗値が$R$の抵抗器の両端にかかる電圧を$V_R(t)$, 流れる電流を$I(t)$とすると, オームの法則により次式が成り立つ.
\begin{equation}
V_R(t) = RI(t)
\end{equation}
コンデンサは充電時間を除けば絶縁体である.
コンデンサの静電容量を$C$とすると, ある時間にコンデンサに蓄えられている電荷$q(t)$とコンデンサの両端にかかる電圧$V_C(t)$の関係は次式で表される.
\begin{equation}
  q(t) = CV_C(t)
\end{equation}
電流とは単位時間あたりに流れる電荷の量であるため, 次式が成り立つ.
\begin{equation}
  I(t) = \frac{dq(t)}{dt}
  \label{eq:current_charge}
\end{equation}
コイルは時間変化する電流に対して誘導起電力を発生させる.
コイルの自己インダクタンスを$L$, ある時間にコイルを流れる電流を$I(t)$とすると, 誘導起電力$V_L(t)$は次式で表される.
\begin{equation}
  V_L(t) = -L \frac{dI(t)}{dt}
\end{equation}

\begin{figure}[H]
  \centering
  \includegraphics[width=0.6\linewidth]{picture/picture_principle.png}
  \caption{RLC直列回路の模式図}
  \label{fig:RLC-series-circuit}
\end{figure}

図\ref{fig:RLC-series-circuit}の回路に電池をつないだ瞬間に回路を流れる電流を考える.

電池の起電力を$V$とし, ある時刻にコンデンサに蓄えられている電荷を$q(t)$, 抵抗値, インでクタンス, コンデンサの静電容量をそれぞれ$R$, $L$, $C$とすると, 回路に流れる電流$I(t)$は次式で表される.
\begin{equation}
  V - L \frac{dI(t)}{dt} = RI(t) + \frac{q(t)}{C}
\end{equation}
両辺を時間で微分し, 式\ref{eq:current_charge}を用いて, $2\gamma=R/L$, $\omega_0^2 = 1/(LC)$とおくと, 次式が得られる.
\begin{equation}
  \frac{d^2I(t)}{dt^2} + 2 \gamma\frac{dI(t)}{dt} + \omega_0^2 I(t) = 0
\end{equation}

この式は$I(t)$を$x(t)$と置き換えると, 減衰振動を表す微分方程式と同じ形になり, 解は$\gamma$と$\omega_0$の大小関係によって次の3通りに分類される.

\vskip\baselineskip
(i) $\gamma < \omega_0$(減衰振動)
\begin{equation}
  I(t) = e^{-\gamma t} (a \sin{\omega_1 t} + b \cos{\omega_1 t})
\end{equation}

(ii) $\gamma > \omega_0$(過減衰)
\begin{equation}
  I(t) = e^{-\gamma t} (ae^{\omega_1 t} + be^{-\omega_1 t})
\end{equation}

(iii) $\gamma = \omega_0$(臨界減衰)
\begin{equation}
  I(t) = e^{-\gamma t} (a t + b)
\end{equation}

ここで, $a$, $b$はいずれの場合も初期条件によって決まる定数, 角周波数$\omega_1$は過減衰の場合$\omega_1 = \sqrt{\gamma^2 - \omega_0^2}$, 減衰振動の場合$\omega_1 = \sqrt{\omega_0^2 - \gamma^2}$である.
実際に図\ref{fig:RLC-series-circuit}の回路を組み立て, 時間と電流の関係を観測すると, 減衰振動, 過減衰, 臨界減衰のいずれの現象も確認でき, 角周波数$\omega_1$や抵抗係数$\gamma$を求めることができる.



\section{実験方法}

直流回路において臨界減衰を観測することは困難であるため, 本実験では図\ref{fig:experimental-method}の回路を用いて減衰振動と過減衰について実験を行った.

回路は$R_0$の抵抗を用いた際に臨界減衰が起こるようなインダクタンス$L$と静電容量$C$を選び, 抵抗値$R$を変えることで減衰振動や過減衰が起こるようにした.
回路の電源として発振器を接続し, 電流の時間変化を直接観測するのではなく, 抵抗感電圧$V_R(t)$をオシロスコープで観測した.

\vskip\baselineskip
\begin{figure}[H]
  \centering
  \includegraphics[width=0.8\linewidth]{picture/picture_method.png}
  \caption{減衰振動, 過減衰の実験回路}
  \label{fig:experimental-method}
\end{figure}

はじめに, コイルのインでクタンス$L$をLCRメータで, コンデンサの静電容量$C$と抵抗値$R$をマルチメータで測定した.
次に, 次式を用いて, 抵抗値の平均$R_0$と$L$から, 臨界減衰($\gamma = \omega_0$)が起こるコンデンサの静電容量$C_0$を計算した.
$2\gamma = R/L$, $\omega_0^2 = 1/(LC)$, $\gamma = \omega_0$より,
\begin{eqnarray}
  \frac{R}{2L} &=& \sqrt{\frac{1}{LC_0}} \nonumber \\
  \frac{R^2}{4L^2} &=& \frac{1}{LC_0} \nonumber \\
  C_0 &=& \frac{4L}{R^2} \label{eq:critical-capacitance}
\end{eqnarray}
測定した静電容量$C$が$C_0$に最も近い値になるようにコンデンサを選んだ.

これらのコイルとコンデンサ, 抵抗器を用い, 与えられた4つの抵抗を順番に繋いで回路を組み立て測定を行った.
\newpage



\section{実験結果}



\subsection{回路定数}

コイルのインダクタンス$L$と抵抗値$R_L$の測定結果を表\ref{tb:measure-result}に, 与えられた4つの抵抗の抵抗値の測定結果を表\ref{tb:resistor-values}に示す.

\begin{table}[H]
  \begin{minipage}{0.48\linewidth}
    \centering
    \caption{コイルのインダクタンスと\protect\linebreak 抵抗値の測定結果}
    \label{tb:measure-result}
    \begin{tabular}{cccc}
      \hline
      インダクタンス & 抵抗値 \\
      $L/\mathrm{mH}$ & $R_L/\Omega$ \\
      \hline
      27.44 & 160.9 \\
      \hline
    \end{tabular}
  \end{minipage}
  \begin{minipage}{0.48\linewidth}
    \centering
    \caption{与えられた抵抗器の\protect\linebreak 抵抗値の測定結果}
    \label{tb:resistor-values}
    \begin{tabular}{cc}
      \hline
      抵抗器番号 & 抵抗値 \\
      & $R/\Omega$ \\
      \hline
      $R_1$ & 130.3 \\
      $R_2$ & 460.8 \\
      $R_3$ & 4697 \\
      $R_4$ & 8045 \\
      \hline
      平均値 & 3333 \\
      \hline
    \end{tabular}
  \end{minipage}
\end{table}


これらの抵抗値の平均値$R_0 = 3333 \Omega$とインダクタンス$L = 27.44 \mathrm{mH}$を用いて, 臨界減衰が起こるコンデンサの静電容量$C_0$を式\ref{eq:critical-capacitance}より計算した.
\begin{equation*}
  C_0 = \frac{4L}{R_0^2} = \frac{4 \times 27.44 \times 10^{-3}}{(3333)^2} \approx 9.88 \times 10^{-9} \mathrm{F} = 9.88 \mathrm{nF}
\end{equation*}

次に, 測定したコンデンサの静電容量$C$を表\ref{tb:capacitor-values}に示す.

\begin{table}[H]
  \centering
  \caption{測定したコンデンサの静電容量}
  \label{tb:capacitor-values}
  \begin{tabular}{cc}
    \hline
    コンデンサ番号 & 静電容量 \\
    & $C/\mathrm{nF}$ \\
    \hline
    $C_1$ & 10.24 \\
    $C_2$ & 2.170 \\
    $C_3$ & 22.22 \\
    $C_4$ & 2.670 \\
    $C_5$ & 3.288 \\
    $C_6$ & 0.404 \\
    $C_7$ & 4.770 \\
    $C_8$ & 6.760 \\
    $C_9$ & 48.13 \\
    $C_{10}$ & 1.425 \\
    \hline
  \end{tabular}
\end{table}

これらの中から, 臨界減衰が起こる静電容量$C_0 = 9.88 \mathrm{nF}$に最も近い値である$C_1 = 10.24 \mathrm{nF}$のコンデンサを選んだ.



\subsection{減衰振動}



\subsubsection{$R_1$の測定結果}

$R_1$で回路を組み立て測定した結果, 減衰振動が観測された.
オシロスコープで観測した波形を図\ref{fig:R1-oscilloscope}に示す.

\begin{figure}[H]
  \centering
  \includegraphics[width=0.8\linewidth]{picture/picture_waveform3.png}
  \caption{$R_1$で観測したオシロスコープの波形}
  \label{fig:R1-oscilloscope}
\end{figure}

この波の各山と谷のピーク電圧$V_n$と1番目の山からの時間$t_n$を測定し, 周期$T_i$や電圧比の対数$\ln{V_i/V_1}$を求めた結果を表\ref{tb:R1-results}に示す.

\begin{table}[H]
  \centering
  \caption{$R_1$での減衰振動の測定結果}
  \label{tb:R1-results}
  \begin{tabular}{ccccc}
    \hline
    山番号 & 時間 & 電圧 & 電圧の絶対値 & 電圧比の対数 \\
    $i$ & $t_i/${\textmu}s & $V_i/\mathrm{V}$ & $|V_i|/\mathrm{V}$ & $\ln{|V_i|/V_1}$ \\
    \hline
    1 & 0 & 0.472 & 0.472 &  0.00000 \\
      & 52 & -0.328 & 0.328 & -0.15807 \\
    2 & 104 & 0.232 & 0.232 & -0.30845 \\
      & 158 & -0.168 & 0.168 & -0.44863 \\
    3 & 212 & 0.112 & 0.112 & -0.62472 \\
      & 266 & -0.08 & 0.080 & -0.77085 \\
    4 & 320 & 0.056 & 0.056 & -0.92575 \\
      & 374 & -0.032 & 0.032 & -1.16879 \\
    \hline
  \end{tabular}
\end{table}

\newpage

次に, 表\ref{tb:R1-results}のデータを用いて, 減衰振動の周期の平均値$\overline{T}$と抵抗係数と回路定数から計算した周期$T_0$を表\ref{tb:R1-period}に示す.

\begin{table}[H]
  \centering
  \caption{$R_1$での減衰振動の周期}
  \label{tb:R1-period}
  \begin{tabular}{cc}
    \hline
    測定周期 $\overline{T}/${\textmu}s & 計算周期 $T_0/${\textmu}s \\
    \hline
    106.7 & 105.7 \\
    \hline
  \end{tabular}
\end{table}

次に, 表\ref{tb:R1-results}のデータを用いて, 横軸に時間$t_i$, 縦軸に電圧比の対数$\ln{V_i/V_1}$をプロットしたグラフを図\ref{fig:R1-graph}に示す.
近似直線の傾きの大きさは-0.006955{\textmu}s$^{-1}$であった.

\begin{figure}[H]
  \centering
  \includegraphics[width=0.8\linewidth]{graph/R1.png}
  \caption{$R_1$での時間と電圧比の対数の関係}
  \label{fig:R1-graph}
\end{figure}

\newpage



\subsubsection{$R_2$の測定結果}

$R_2$で回路を組み立て測定した結果, 減衰振動が観測された.
オシロスコープで観測した波形を図\ref{fig:R2-oscilloscope}に示す.

\begin{figure}[H]
  \centering
  \includegraphics[width=0.8\linewidth]{picture/picture_waveform2.png}
  \caption{$R_2$で観測したオシロスコープの波形}
  \label{fig:R2-oscilloscope}
\end{figure}

この波の各山と谷のピーク電圧$V_n$と1番目の山からの時間$t_n$を測定し, 周期$T_i$や電圧比の対数$\ln{V_i/V_1}$を求めた結果を表\ref{tb:R2-results}に示す.

\begin{table}[H]
  \centering
  \caption{$R_2$での減衰振動の測定結果}
  \label{tb:R2-results}
  \begin{tabular}{ccccc}
    \hline
    山番号 & 時間 & 電圧 & 電圧の絶対値 & 電圧比の対数 \\
    $i$ & $t_i/${\textmu}s & $V_i/\mathrm{V}$ & $|V_i|/\mathrm{V}$ & $\ln{|V_i|/V_1}$ \\
    \hline
    1 & 0 & 1.46 & 1.46 &  0.00000 \\
      & 53 & -0.74 & 0.74 & -0.29512 \\
    2 & 106 & 0.38 & 0.38 & -0.58457 \\
      & 159 & -0.20 & 0.20 & -0.86332 \\
    3 & 212 & 0.10 & 0.10 & -1.16435 \\
    \hline
  \end{tabular}
\end{table}

\newpage

次に, 表\ref{tb:R2-results}のデータを用いて, 減衰振動の周期の平均値$\overline{T}$と抵抗係数と回路定数から計算した周期$T_0$を表\ref{tb:R2-period}に示す.

\begin{table}[H]
  \centering
  \caption{$R_2$での減衰振動の周期}
  \label{tb:R2-period}
  \begin{tabular}{cc}
    \hline
    測定周期 $\overline{T}/${\textmu}s & 計算周期 $T_0/${\textmu}s \\
    \hline
    106.0 & 107.3 \\
    \hline
  \end{tabular}
\end{table}

次に, 表\ref{tb:R2-results}のデータを用いて, 横軸に時間$t_i$, 縦軸に電圧比の対数$\ln{V_i/V_1}$をプロットしたグラフを図\ref{fig:R2-graph}に示す.
近似直線の傾きの大きさは-0.012586{\textmu}s$^{-1}$であった.

\begin{figure}[H]
  \centering
  \includegraphics[width=0.8\linewidth]{graph/R2.png}
  \caption{$R_2$での時間と電圧比の対数の関係}
  \label{fig:R2-graph}
\end{figure}

\newpage



\subsection{過減衰}



\subsubsection{$R_3$の測定結果}

$R_3$で回路を組み立て測定した結果, 過減衰が観測された.
オシロスコープで観測した波形を図\ref{fig:R3-oscilloscope}に示す.

\begin{figure}[H]
  \centering
  \includegraphics[width=0.8\linewidth]{picture/picture_waveform1.png}
  \caption{$R_3$で観測したオシロスコープの波形}
  \label{fig:R3-oscilloscope}
\end{figure}

この波の各電圧$V_n$と1番目の山からの時間$t_n$を測定し, 電圧比の対数$\ln{V_i/V_1}$を求めた結果を表\ref{tb:R3-results}に示す.

\begin{table}[H]
  \centering
  \caption{$R_3$での過減衰の測定結果}
  \label{tb:R3-results}
  \begin{tabular}{ccc}
    \hline
    時間 & 電圧 & 電圧比の対数 \\
    $t_i/${\textmu}s & $V_i/\mathrm{V}$ & $\ln{|V_i|/V_1}$ \\
    \hline
    0 & 5.52 & 0.00000 \\
    25 & 3.44 & -0.20538 \\
    50 & 1.92 & -0.45864 \\
    75 & 1.12 & -0.69272 \\
    100 & 0.64 & -0.93576 \\
    125 & 0.40 & -1.13988 \\
    \hline
  \end{tabular}
\end{table}

\newpage

次に, 表\ref{tb:R3-results}のデータを用いて, 横軸に時間$t_i$, 縦軸に電圧比の対数$\ln{V_i/V_1}$をプロットしたグラフを図\ref{fig:R3-graph}に示す.
近似直線の傾きの大きさは-0.021380{\textmu}s$^{-1}$であった.

\begin{figure}[H]
  \centering
  \includegraphics[width=0.8\linewidth]{graph/R3.png}
  \caption{$R_3$での時間と電圧比の対数の関係}
  \label{fig:R3-graph}
\end{figure}

\newpage



\subsubsection{$R_4$の測定結果}

$R_4$で回路を組み立て測定した結果, 過減衰が観測された.
オシロスコープで観測した波形を図\ref{fig:R4-oscilloscope}に示す.

\begin{figure}[H]
  \centering
  \includegraphics[width=0.8\linewidth]{picture/picture_waveform4.png}
  \caption{$R_4$で観測したオシロスコープの波形}
  \label{fig:R4-oscilloscope}
\end{figure}

この波の各電圧$V_n$と1番目の山からの時間$t_n$を測定し, 電圧比の対数$\ln{V_i/V_1}$を求めた結果を表\ref{tb:R4-results}に示す.

\begin{table}[H]
  \centering
  \caption{$R_4$での過減衰の測定結果}
  \label{tb:R4-results}
  \begin{tabular}{ccc}
    \hline
    時間 & 電圧 & 電圧比の対数 \\
    $t_i/${\textmu}s & $V_i/\mathrm{V}$ & $\ln{|V_i|/V_1}$ \\
    \hline
    0 & 6.56 & 0.00000 \\
    25 & 4.72 & -0.14296 \\
    50 & 3.52 & -0.27036 \\
    75 & 2.56 & -0.40866 \\
    100 & 1.92 & -0.53360 \\
    125 & 1.44 & -0.65854 \\
    150 & 1.04 & -0.79987 \\
    175 & 0.80 & -0.91381 \\
    200 & 0.56 & -1.06872 \\
    225 & 0.40 & -1.21484 \\
    250 & 0.32 & -1.31175 \\
    \hline
  \end{tabular}
\end{table}

\newpage

次に, 表\ref{tb:R4-results}のデータを用いて, 横軸に時間$t_i$, 縦軸に電圧比の対数$\ln{V_i/V_1}$をプロットしたグラフを図\ref{fig:R4-graph}に示す.
近似直線の傾きの大きさは-0.012156{\textmu}s$^{-1}$であった.

\begin{figure}[H]
  \centering
  \includegraphics[width=0.8\linewidth]{graph/R4.png}
  \caption{$R_4$での時間と電圧比の対数の関係}
  \label{fig:R4-graph}
\end{figure}

\newpage



\subsection{回路定数から求めた値との比較}

次に回路定数から, $\gamma$(減衰振動)と$\gamma-\omega_1$(過減衰)の値を計算し, 減衰振動と過減衰でそれぞれ求めた近似直線の傾きの大きさと比較した結果を表\ref{tb:comparison}, 表\ref{tb:final-comparison}に示す.
ただし, この計算で用いる抵抗値を$R'$とすると, 測定した抵抗値$R_1$, $R_2$, $R_3$, $R_4$それぞれに, コイルの内部抵抗$R_L$(ここでは$160.9 \Omega$)と発振器の出力インピーダンス$50 \Omega$を加える必要がある.
また, $\omega_0=1/\sqrt{1/(LC)}=5.966\times10^4\mathrm{rad}/\mathrm{s}$である.

\begin{table}[H]
  \centering
  \caption{回路定数から求めた$\gamma$, $\gamma - \omega_1$の値}
  \label{tb:comparison}
  \begin{tabular}{ccccc}
    \hline
    抵抗器番号 & 抵抗値 & $\gamma$ & $\omega_1$ & $\gamma - \omega_1$ \\
    & $R'/\Omega$ & $\Omega/\mathrm{H}$ & $\mathrm{rad}/\mathrm{s}$ & \\
    \hline
    $R_1$ & 341.19 & 6.217$\times10^3$ & 5.933$\times10^4$ & - \\
    $R_2$ & 671.70 & 1.224$\times10^4$ & 5.839$\times10^4$ & - \\
    $R_3$ & 4907.9 & 8.943$\times10^4$ & 6.662$\times10^4$ & 2.281$\times10^4$ \\
    $R_4$ & 8255.9 & 1.504$\times10^5$ & 1.381$\times10^5$ & 1.233$\times10^4$ \\
    \hline
  \end{tabular}
\end{table}

\begin{table}[H]
  \centering
  \caption{回路定数から求めた値と近似直線の傾きの比較}
  \label{tb:final-comparison}
  \begin{tabular}{cccc}
    \hline
    抵抗器番号 & 回路定数から求めた値 & 近似直線の傾き \\
    & $-\gamma$ or $-(\gamma - \omega_1)$ & $s^{-1}$ \\
    \hline
    $R_1$ & -6.217$\times10^3$ & -6.955$\times10^3$ \\
    $R_2$ & -1.224$\times10^4$ & -1.259$\times10^4$ \\
    $R_3$ & -2.281$\times10^4$ & -2.138$\times10^4$ \\
    $R_4$ & -1.233$\times10^4$ & -1.216$\times10^4$ \\
    \hline
  \end{tabular}
\end{table}



\section{考察}



\subsection{減衰振動と過減衰の観測}

今回, 抵抗器$R_1$と$R_2$を用いた際に減衰振動が観測され, 抵抗器$R_3$と$R_4$を用いた際に過減衰が観測された.
これは, 抵抗値が小さいほど抵抗係数$\gamma$が小さくなり, 角$\gamma < \omega_0$の条件を満たすためである.
逆に, 抵抗値が大きいほど抵抗係数$\gamma$が大きくなり, $\gamma > \omega_0$の条件を満たすためである.
したがって, 抵抗値を変えることで, 減衰振動と過減衰の両方の現象を観測できたと考えられる.

\subsection{回路定数から求めた値との比較}

表\ref{tb:final-comparison}に示したように, 回路定数から求めた値と近似直線の傾きは, いずれの場合もほぼ一致していると言える.
これは, 減衰振動や過減衰の理論式が正しく成り立っていることを示していると考えられる.
また, 減衰振動の場合, 抵抗器$R_1$よりも$R_2$の方が回路定数から求めた値と近似直線の傾きの差が小さくなっている.
これは, 抵抗値が大きくなるほど, 抵抗係数$\gamma$が大きくなり, 減衰の影響が強くなるため, 測定誤差の影響が相対的に小さくなることが原因であると考えられる.
一方, 過減衰の場合, 抵抗器$R_3$よりも$R_4$の方が回路定数から求めた値と近似直線の傾きの差が小さくなっている.
これは, 抵抗値が大きくなるほど, 抵抗係数$\gamma$が大きくなり, 減衰の影響が強くなるため, 測定誤差の影響が相対的に小さくなることが原因であると考えられる.




\section{まとめ}

本実験では, 抵抗器, コイル, コンデンサを用いてRLC直列回路を組み立て, 減衰振動と過減衰の現象を観測した.
抵抗値を変えることで, 減衰振動と過減衰の両方の現象を確認できた.
また, 減衰振動の周期や抵抗係数, 過減衰の抵抗係数から角周波数を求めることができた.



% \begin{thebibliography}{99}


% \end{thebibliography}

\end{document}